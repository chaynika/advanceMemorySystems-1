\documentclass[conference]{IEEEtran}
\IEEEoverridecommandlockouts
% The preceding line is only needed to identify funding in the first footnote.
% If that is unneeded, please comment it out.
\usepackage{cite}
\usepackage{amsmath,amssymb,amsfonts}
\usepackage{algorithmic}
\usepackage{graphicx}
\usepackage{textcomp}
\def\BibTeX{{\rm B\kern-.05em{\sc i\kern-.025em b}\kern-.08em
    T\kern-.1667em\lower.7ex\hbox{E}\kern-.125emX}}
\begin{document}

\title{Proposal\\
Blue-Collar Node : A Sharding Approach for Optimizing Storage on Blockchains
% Sharding for Blockchains Memory Optimization
% {\footnotesize \textsuperscript{*}Note: Sub-titles are not captured in Xplore and
% should not be used}
% \thanks{Identify applicable funding agency here. If none, delete this.}
}

\author{\IEEEauthorblockN{1\textsuperscript{st} Chawla}
\IEEEauthorblockA{\textit{Computer Science} \\
\textit{Arizona State University}\\
nchawla3@asu.edu}
\and
\IEEEauthorblockN{2\textsuperscript{nd} Saikia}
\IEEEauthorblockA{\textit{Computer Engineering} \\
\textit{Arizona State University}\\
csaikia@asu.edu}
}

\maketitle

\begin{abstract}
    The scalability of blockchain is a primary and urgent concern. The current
    popular blockchains have fundamental bottlenecks which limit their ability
    to have a higher throughput and a lower latency. One of the bottlenecks is
    the storage requirements of the current blockchains. All the full nodes and
    miners in a blockchain are required to store the complete blockchains and it
    grows everytime more transactions are added. The network relies mainly on
    commodity hardware for block propagation and transaction verification.
    With the increase in the size of blockchains, storing the entire blockchain
    on voluntary full nodes becomes impractical. Our project focuses at sharding
    this blockchain and storing it in several nodes. There is a two-fold advantage of this,
    transactions can be verified in parallel by different nodes and consensus
    can be achieved in a faster way increasing the throughput of the network,
    and the storage requirement of the full node will be decreased
    substantially. That directly will prevent the blockchain from getting
    centralized towards supercomputers. However, in this project we will focus
    only on the storage part of the problem. Executing transaction in parallel
    will be a part of the future work.

\end{abstract}

\begin{IEEEkeywords}
    blockchain, storage, sharding 
\end{IEEEkeywords}

\section{Introduction}


\section{Background And Related Work}

\subsection{Background}
    Blockchain is a decentralized, tamper-proof digital ledger technology that
    is transforming transaction prospects for many industries. In addition to
    being the foundation of cryptocurrencies like Ethereum, the blockchain
    structure, which consists of linked blocks of information, allows for direct
    transactions across a network of computers without need for a central
    authority. The Ethereum network consists of three kind of nodes: full nodes,
    light-weight nodes and miners. Full nodes saves the complete blockchain and
    has two main functions: to validate a transaction and to propagate the data
    across the network. Light-weight nodes are used only to propagate the
    transactions or blocks, they do not validate the transaction. The validation
    part here is minimal and is done by SPV or by downloading random chunks of
    a block and verify them using their merkle roots. Miners store the entire
    blockchain so that all transactions can be verified and create new block
    that can be added to the blockchain. 
    Since full nodes store the entire data blockchain, the storage requirement
    is ever increasing. The current bitcoin blockchain is around 120GB and
    ethereum blockchain is 100+ GB. As the system is accepted more, the need for
    storage will increase and only nodes that have a higher configuration will
    be able to store this data. This leads to centralizing of the blockchain
    network. In order to prevent this kind of centralization, we through this
    project want to shard and store the blockchain into many nodes that are
    connected such that the data can be 


\subsection{Related Work}


\section{Proposed Solution}
\subsection{Components}
The key component of our system will be what we are calling a "Blue-collar node".
This node will be a combination of a light and full-node. This node will be
reuqired to store the sharded blockchain and the parity bits such that if needed
it can re-create the data from a connected full node. 
Data-distribution algorithm : 
Monitor : 

\subsection{Architecture}

\subsection{Implementation Plan}

\subsection{Evaluation Plan}
The proposed system would then be evaluated as to how much maximum storage is
required per node after we have sharded the blockchain. A ratio will be a good
metric to validate if the solution achieved what it was intended to achieve.
We can utilize the moore's law to predict the size of blockchain, average HDD
and cost of the hardware to figure out if the amount of reduction in storage
achieved by our system will be good enough.

\section{Proposed Schedule}
\begin{tabular}{|c||c||c|}
    \hline
    \textbf{Task} & \textbf{Begin date} & \textbf{End date} \\
    \hline
    0 & 1 & 2 \\
    \hline
\end{tabular}

\cite{b1}


\begin{thebibliography}{00}
\bibitem{b1} G. Eason, B. Noble, and I. N. Sneddon, ``On certain integrals of Lipschitz-Hankel type involving products of Bessel functions,'' Phil. Trans. Roy. Soc. London, vol. A247, pp. 529--551, April 1955.
\end{thebibliography}

\end{document}
